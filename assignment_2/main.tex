\documentclass{article}
\usepackage[utf8]{inputenc}
\usepackage[utf8, left=2.5cm, right=2.5cm, top=2.5cm, bottom=2.5cm]{geometry}
\usepackage[english]{babel}
\setlength{\parindent}{2em}
\setlength{\parskip}{1em}
\renewcommand{\baselinestretch}{1}
\usepackage{graphicx}
\usepackage{gensymb}
\usepackage{natbib}
\usepackage{amssymb}
\usepackage{xcolor}


\title{CTA200 Assignment 2 Question 4}
\author{Patricia Golaszewska}
\date{May $8^{th}$ 2021}
\begin{document}

\maketitle

\newpage

\begin{center}
    \section*{Question 1}
\end{center}

\section*{Methods}

In this problem, we aim to analyze two methods of differentiation. The first method is often referred to as the forward difference method, and the second as the centered difference method. I began by computing the value of the derivative of the function $sin(x)$ for h values of 0.001, 0.01, and 0.1 to establish a baseline of the accuracy. From there, I wanted to introduce the analytical derivative of $sin(x)$ and contrast the error between the analytical and numerical approaches. In order to do this, I needed to modify the original equations for the forward and centered difference slightly. They would now take a range of h values, versus feeding them individual values. From there, I could create a function which would illustrate the error between the analytical result and the forward difference method, and another which illustrated the error between the analytical result and the centered difference method. Afterwards, I plotted the error on a loglog plot, and Figure 1 was obtained. 


\section*{Analysis}

In reference to the graph presented (Figure 1), the error between the two methods and analytical solution are presented. This graph is a loglog plot, generated as the value of h varies. The slope of each line indicates the growth of error between the methods and the analytical solution. As the value of h approaches zero, the error is much smaller. But as the value of h (the separation distance) gets larger, the error of the centered difference grows at a faster rate than that of the forward difference. I assume that this is because of the difference in the dependence oh h in both methods. In regards to evaluating the numerical derivative for different values of h, we obtain a more accurate value for the derivative.   



\begin{figure}[htp]
    \centering
    \includegraphics[width=12cm]{Q1.pdf}
    \caption{Error for centered and forward difference methods}
    \label{fig:Q1}
\end{figure}

\newpage

\begin{center}
    \section*{Question 2}
\end{center}

\section*{Methods}

In this problem, we are working with points in the complex plane. I first started by defining an array of x and y values in the specified range. I could then create the points in the complex plane, c. From there, we could then iterate the Mandelbrot set, given that we first created a mesh grid for a and y. An image was generated by filtering the z values and limiting them. Had I provided a longer set of iteration numbers, we would have seen a more defined image. We expect values that do not diverge, and so they must be bounded to not exceed infinity. We can see from the image that the unique boundary of the Mandelbrot set is generated, and the classic shape is obtained. Admittedly, I ran out of time to complete this assignment. Had I had more time, I would have generated another map.

\section*{Analysis}

Studies in nonlinear dynamics and chaos show us the relationship between the Mandelbrot set and the bifurcation diagram of the logistic map. Our points are contained in a closed and compact set. Our points are generated by a quadratic recurrence equation, and the iteration of it in our code provides us with exactly the result we expect. Because the solution is meant to be bounded, there is some value for which the iterations cannot be larger than. Our simple formula gives us a very complex result, and our graph provides a visual representation of how quickly our values can diverge.

\begin{figure}[htp]
    \centering
    \includegraphics[width=12cm]{Q2.pdf}
    \caption{Mandelbrot Set}
    \label{fig:Q2}
\end{figure}

\newpage

\begin{center}
    \section*{Question 3}
\end{center}

\section*{Methods}

In order to solve this system of ordinary differential equations, I chose to employ the dopri5 integrator. In order to solve our problem, I needed to first create a function which contained our equations. From there, I was able to feed that function into an integrator. In order to produce numerical results, I needed to provide the integrator with a set of initial conditions and function parameters. Values were chosen for $\beta$, $N$, and $\gamma$, and the integration process could begin. First, an empty array was created as a means of storing the solution to the numerical integral, and a counter was set up. This counter would allow for us to store a solution for each point we passed through as we took the integral. From there, we could plot the solutions for each variable (S, I, or R) on the same graph. This process was repeated 3 times for the SIR model. We then introduced a fourth variable for D, and adapted the model to include death. This model was then solved in a similar fashion, but a 4-D approach was taken versus a 3-D approach.

\section*{Analysis}

In the models, values of $\beta$ and $\gamma$ were chosen to reflect a likely transmission and infection pathway. The variable $\gamma$ represents the recovery rate of people infected with this disease. This implies that $\gamma$ is bounded between 0 and 1. If $\gamma$ is 1, this means that every individual will recover, versus 0, where no one will recover. Similar conditions apply to $\beta$; not every interaction is guaranteed to result in disease transmission, but there has to be some transmission taking place. In all models, we see that the number of susceptible people will drop, and won't return to the initial value. This is because these individuals now have immunity. The number of infected people will fluctuate, as these are the individuals infected at a given time. The recovery rate differs in the SIR models versus the SIRD model, as now the infected individuals may succumb to their disease rather than recover. 

\begin{figure}[htp]
    \centering
    \includegraphics[width=12cm]{Q3.1.pdf}
    \caption{$\beta$ = 0.5 and $\gamma$ = 0.2}
    \label{fig:Q3.1}
\end{figure}


\begin{figure}[htp]
    \centering
    \includegraphics[width=12cm]{Q3.2.pdf}
    \caption{$\beta$ = 0.8 and $\gamma$ = 0.09}
    \label{fig:Q3.2}
\end{figure}


\begin{figure}[htp]
    \centering
    \includegraphics[width=12cm]{Q3.3.pdf}
    \caption{$\beta$ = 0.7 and $\gamma$ = 0.1}
    \label{fig:Q3.3}
\end{figure}


\begin{figure}[htp]
    \centering
    \includegraphics[width=12cm]{Q3.4.pdf}
    \caption{$\beta$ = 0.8, $\gamma$ = 0.3, and $\mu$ = 0.05}
    \label{fig:Q3.4}
\end{figure}

\end{document}
